\documentclass[12pt]{article}
\usepackage[utf8]{inputenc}
\usepackage{graphicx}
\usepackage{hyperref}
\usepackage{geometry}
\geometry{margin=1in}
\title{BookNook: Online Book Store\\\large Semester Project Report}
\author{
Mudassar Abbas (01-135231-052)\\
Muhammad Aleem (01-135231-055)\\
BS(IT) – 5B\\
Bahria University, Islamabad Campus
}
\date{\today}

\begin{document}

\maketitle

\section*{1. Introduction}
This document presents the final report for our semester project ``BookNook,'' an online book store developed using HTML, CSS, and JavaScript. It allows users to browse, search, and purchase books online with ease.

\section*{2. Previous SRS Summary}
\textbf{Title:} BookNook – Your Online Book Store

\textbf{Overview:} BookNook is a user-friendly e-commerce application enabling book browsing, searching, purchasing, and order tracking. It features admin functionalities and responsive design.

\textbf{Key Features:}
\begin{itemize}
    \item Browse books by categories
    \item Search by title, author, or genre
    \item Shopping cart and checkout
    \item User authentication
    \item Admin dashboard
    \item Mobile responsiveness
\end{itemize}

\textbf{Development Stack:}
\begin{itemize}
    \item Frontend: React.js (planned), Tailwind CSS
    \item Backend: ASP.NET + Node.js with Express (planned)
    \item Database: MongoDB
    \item Tools: JWT, Stripe, Cloudinary
\end{itemize}

\section*{3. Final Front-End Structure}
The front-end was built using:
\begin{itemize}
    \item HTML5 – for structure
    \item CSS3 – for styling and layout
    \item JavaScript – for interactivity (navigation, cart functionality)
\end{itemize}

The site comprises:
\begin{itemize}
    \item \textbf{Home Page:} Showcases featured books and categories
    \item \textbf{Books Listing Page:} Displays books by category
    \item \textbf{Book Detail Page:} Shows details and "Add to Cart"
    \item \textbf{Cart Page:} Manages selected items
    \item \textbf{Login/Signup Page:} Basic form design
\end{itemize}

\section*{4. Screenshots}
\begin{figure}[h]
\centering
\includegraphics[width=0.8\textwidth]{screenshots/home.png}
\caption{Home Page}
\end{figure}

\begin{figure}[h]
\centering
\includegraphics[width=0.8\textwidth]{screenshots/book-details.png}
\caption{Book Detail Page}
\end{figure}

\section*{5. Connection to Backend}
Currently, the project includes only front-end components. Backend integration was planned using Node.js/Express with MongoDB, but due to time limitations, backend API and payment functionality were not implemented. JSON-based placeholder data was used to simulate functionality.

\section*{6. Challenges and Resolutions}
\begin{itemize}
    \item \textbf{Challenge:} Delay in backend development.
    \item \textbf{Resolution:} Simulated backend using JavaScript arrays/objects for product data.
    \item \textbf{Challenge:} Responsive design issues on small devices.
    \item \textbf{Resolution:} Manual media queries and flexbox adjustments.
    \item \textbf{Challenge:} Cart item persistence.
    \item \textbf{Resolution:} Used browser \texttt{localStorage}.
\end{itemize}

\section*{7. Deviations from Proposal}
The original proposal planned to use:
\begin{itemize}
    \item React.js for dynamic UI
    \item Tailwind CSS for styling
    \item ASP.NET/Node.js backend
\end{itemize}

However, due to time constraints, we developed the frontend using:
\begin{itemize}
    \item Plain HTML, CSS, and basic JavaScript
    \item No backend or database integration
    \item No payment gateway integration
\end{itemize}

\section*{8. Conclusion}
BookNook's front-end development is complete, showcasing a clean and responsive UI. Core e-commerce functionalities are implemented in static form, and we aim to integrate a backend and full user experience in the future.

\end{document}

